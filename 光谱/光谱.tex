\documentclass[12pt]{article}

\usepackage{ctex}

\usepackage{graphicx,float,indentfirst,amsmath,amssymb,geometry,subfig,hyperref,tikz}
\usepackage{bookmark}
\usetikzlibrary{arrows.meta, calc}

\hypersetup{hidelinks}

\geometry{a4paper,scale=0.8}

\title{光谱实验报告}
\author{\kaishu 工物42 \quad \kaishu 李中翔 \quad \kaishu 2024011032}
\date{\today}
\pagenumbering{arabic}

\begin{document}

\maketitle
\tableofcontents
\newpage
\section{摘要}

复色光经过色散系统分光后单色光将按照波长大小依次排列形成光谱,在现代科学中我们为了定量分析常常用复色光中各单色光光强随波长变化曲线来记录光谱。我们通过观察分析发射、吸收或反射光谱可以获得物质微观结构、化学成分和运动状态等信息。在本实验中将采用光栅光谱仪这一定量测量光谱成分的精密仪器来对连续谱(如手电筒灯光)和分立谱光源(如氢灯)进行光谱分析,还会进一步探究防晒霜吸收紫外线效果和黑体辐射。

\section{实验原理}
\subsection{光纤光谱仪结构}

    本实验使用Czerny-Turner结构的光纤光谱仪,我们从图1中可以看出被测光经光纤传导从狭缝$S_1$入射后(近似可以看作点光源)经过$M_1$反射(我们将$S_1$放置于凹面镜$M_1$的前焦面)形成平行光入射到光栅G,同一波长同一级次的衍射光经过光栅后仍然是平行光,然后经过物镜$M_2$汇聚于CCD形成单色光的像点。由于不同波长衍射角度不同,使用它们将成像于CCD不同位置,计算机将光强信息读出并以光谱曲线形式呈现出来。
    \begin{figure}[H]
    \centering
    \subfloat{
    \includegraphics[width=0.5\textwidth]{2-1.png}}
    \caption{光栅光谱仪结构}
    \end{figure}

\subsection{光源谱线宽度与相干长度}

    我们无法制造出绝对的单色光源,使用在实验室中使用的单色光源都是以波长$\lambda_0$为中心包含一定波长范围的准单色光,我们如果将光强-波长曲线画出将会发现中心$\lambda_0$处强度最大,在两侧迅速减小,我们把等于最大强度一半的波长范围$\Delta \lambda$称为谱线宽度,而准单色光的谱线宽度与相干长度$\Delta s_c$的关系为$\Delta s_c \Delta \lambda=\lambda_0^2$。

\subsection{元素的特征谱线}

    不同元素原子收到激发时发出的光为线状谱,称之为元素的特征谱线。其中人们对氢原子光谱的研究就推动了量子论的发展,在氢原子光谱中可见区的谱线分布我们可以用公式$\dfrac{1}{\lambda}=R_H(\dfrac{1}{2^2}-\dfrac{1}{n^2})$来表示,其中$R_H$我们称之为氢的里德伯常数。

\subsection{热辐射光谱}

    温度高于绝对零度的物体会发出电磁波,这称为热辐射现象。而还有物体可以完全吸收照射在它上面各种波长光的物体,我们称之为黑体。黑体辐射这一问题困扰了物理学家们许多年,最后普朗克给出了黑体辐射公式使得这一问题得到解决,在本实验中我们将用溴钨灯作为热辐射源,并通过$I(\lambda,T)=\dfrac{A\epsilon(T)}{\lambda^5}\dfrac{1}{e^{C_2/\lambda T}-1}$从实际发射光谱拟合得到温度$T$,其中$\epsilon(T)$是一个与波长无关的常数。此外从普朗克公式还可以得到黑体辐射的总辐出度$M=\sigma T^4$,其中$\sigma$为斯特潘-玻尔兹曼常量,这样我们就可以知道电源的功率$P$正比于$T^4$。

    此外在本实验中我们需要对光谱仪直接测得的光强$I'(\lambda,T)$进行修正,其与入射光的实际光强$I(\lambda,T)$的关系为$I(\lambda,T)=I'(\lambda,T)\xi(\lambda)$,其中$\xi(\lambda)$为传递函数,我们只要求出传递函数就可以对光谱仪测得能量进行修正。
\section{实验仪器}

AMOS型光纤光谱仪、红色LED灯、红色激光器、氢灯、钠灯、汞灯、溴钨灯、电源、玻璃瓶、防晒霜。

\section{实验步骤与数据分析}
\subsection{观察常见光源的光谱}
    我们将把手机屏幕调整为白色,光谱仪的光纤头对准手机屏幕并选择合适的积分时间,采集光谱数据,然后我们以同样的方式采集手机手电筒和红色LED灯的光谱数据。最后我们打开红色激光器并将其发出的激光投射于白色光谱,我们将光纤头对准光屏激光点采集光谱数据。我们将所得光谱数据利用origin软件绘制出光谱曲线。

    \begin{figure}[htbp]
    \begin{minipage}[t]{0.45\linewidth}
    \centering
    \includegraphics[height=4.5cm,width=7cm]{手机屏幕.png}
    \caption{手机屏幕光谱曲线}
    \end{minipage}%
    \begin{minipage}[t]{0.45\linewidth}
    \centering
    \includegraphics[height=4.5cm,width=7cm]{手电筒.png}
    \caption{手机手电筒光谱曲线}
    \end{minipage}
    \end{figure}

    \begin{figure}[htbp]
    \begin{minipage}[t]{0.45\linewidth}
    \centering
    \includegraphics[height=4.5cm,width=7cm]{LED.png}
    \caption{红色LED光谱曲线}
    \end{minipage}%
    \begin{minipage}[t]{0.45\linewidth}
    \centering
    \includegraphics[height=4.5cm,width=7cm]{激光.png}
    \caption{红色激光光谱曲线}
    \end{minipage}
    \end{figure}

    我们可以发现手机屏幕的光谱曲线有三个明显的峰,这是因为本实验中手机屏幕采用的OLED来发光,对于这种屏幕而言,屏幕由无数个微小的、独立的红色、绿色和蓝色子像素组成。所以这三个分离的、较窄的峰值分别对应红色、绿色和蓝色有机发光材料的特征发射峰。而我们发现手机手电筒的光谱曲线由一个高大的窄峰和一个一个宽大的“驼峰”组成,这是因为手机手电筒使用蓝光LED+黄色荧光粉的方式进行发光,所以窄峰对应蓝光LED,而宽大的"驼峰"是因为黄色荧光粉产生的宽谱带发光。

    红色LED和红色激光的光谱曲线都是单峰的,我们通过origin分析可以发现二者的中心波长都大致为$\lambda=636\,nm$,不过二者在谱线宽度上存在差异,红色LED的谱线宽度为$\Delta \lambda_1=18.000\,nm$,因此其相干长度$\Delta s_1=22.472\,\mu m$。红色激光的谱线宽度为$\Delta \lambda_2=1.028\,nm$,因此其相干长度$\Delta s_2\approx 393.479\, \mu m$。而在上学期我们通过迈克尔逊实验得到的红色LED相干长度为$22.500\, \mu m$,二者吻合的很好。

    \subsection{测量元素灯灯特征光谱}

    我们利用光谱仪分别测量氢灯、钠灯和汞灯的光谱得到它们的光谱曲线。
    \begin{figure}[H]
    \centering
    \subfloat{
        \includegraphics[width=0.55\textwidth]{H灯.png}
    }
    \caption{H灯光谱曲线}
    \end{figure}

    

    我们首先观察H灯的光谱曲线,我们发现在可见光范围内有四个峰,它们对应的波长分别为$\lambda_1=410.82\, nm,\lambda_2=434.64\,nm,\lambda_3=487.32\,nm,\lambda_4=656.08\,nm$,这和理论上H原子四个特征谱线是相符的,下面我们利用它们和$\dfrac{1}{\lambda}=R_H(\dfrac{1}{2^2}-\dfrac{1}{n^2})$来计算氢的里德堡常数。

    \begin{table}[ht]
\centering
\caption{H原子可见区的四条特征谱线与里德伯常数}
    \begin{tabular}{|c|c|c|}\hline
    $\lambda/nm$&n&$R_H/m^-1$\\\hline
    410.82  & 6  & $1.095\times 10^7$  \\\hline
    434.64 &5&$1.096\times 10^7$\\\hline
    487.32&4&$1.094\times 10^7$\\\hline
    656.08&5&$1.097\times 10^7$\\\hline
    \end{tabular}%
\end{table}%

    我们取四组计算值的平均值可以得到$R_H=1.0955\times 10^7\, m^{-1}$,这和真实值$1.097\times 10^7\, m^{-1}$非常接近。
    \begin{figure}[H]
    \centering
    \subfloat{
        \includegraphics[width=0.55\textwidth]{Na灯.png}
    }
    \caption{Na灯光谱曲线}
    \end{figure}

    我们再观察Na灯的光谱曲线,我们可以观察到在$\lambda=589.0\,nm$和$\lambda=589.6\,nm$处有两个峰(由于二者非常接近,所以之间没有一个很明显的低谷),这正是Na原子光谱的主线系,我们再向两侧观察可以发现$\lambda=616.2\, nm$处有一个小峰,不过由于和主线系的峰接近所以并不明显,这对应着理论上的锐线系$\lambda=616.0\,nm$和$\lambda=615.4\,nm$。在左侧$\lambda=568.9\,nm$和$\lambda=497.9\,nm$处也有两个峰(都较小),对应着漫线系的$\lambda=568.3\,nm,\lambda=568.8\,nm$和$\lambda=497.78\,nm,\lambda=498.2\,nm$。不过除了$\lambda=589.0\,nm$和$\lambda=589.6\,nm$较大的峰,受限于测量数据的精度,我们无法分辨出其他的双峰。

    \begin{figure}[H]
    \centering
    \subfloat{
        \includegraphics[width=0.55\textwidth]{Hg灯.png}
    }
    \caption{Hg灯光谱曲线}
    \end{figure}

    最后我们再来观察分析Hg灯的特征曲线,我们发现在$\lambda=365.6\,nm,405.2\,nm,436.7\,nm,546.6\,nm,578.7\,nm$处有明显的峰,而这些峰对应的波长和高压汞灯理论上的特征谱线是相符的。

    \subsection{测量介质的光谱透射率}
    将溴钨灯电源电压设置为12V,电流设置为3A,打开溴钨灯并预热3分钟。拨动溴钨灯箱出光孔下方开关去除毛玻璃,让光纤头正对溴钨灯灯丝,调整积分时间使得光谱强度较大但未饱和。我们在透射测量校准后将玻璃片放上记录光谱透射数据,然后取下玻璃片换上涂有防晒霜的玻璃片并记录光谱透射数据。此时我们数据只有可见光部分可信,我们将样品取下,关闭透射测量,调整积分时间使得光谱曲线可见光波长内强度较大的部分谱线饱和而波长低于400nm的未饱和,此时重新进行透射测量校准并测量玻璃片和涂有防晒霜的玻璃片的光谱透射数据。

    我们根据得到的光谱数据可以得到在两种积分时间下玻璃片和涂有防晒霜的玻璃片的光谱透射率曲线。

    \begin{figure}[htbp]
    \begin{minipage}[t]{0.45\linewidth}
    \centering
    \includegraphics[height=4.5cm,width=7cm]{玻璃片(可见光).png}
    \caption{玻璃片透射率曲线(可见光)}
    \end{minipage}%
    \begin{minipage}[t]{0.45\linewidth}
    \centering
    \includegraphics[height=4.5cm,width=7cm]{玻璃片(紫外红外).png}
    \caption{玻璃片透射率曲线(紫外红外)}
    \end{minipage}
    \end{figure}

    \begin{figure}[htbp]
    \begin{minipage}[t]{0.45\linewidth}
    \centering
    \includegraphics[height=4.5cm,width=7cm]{防晒霜(可见光).png}
    \caption{涂防晒霜透射率曲线(可见光)}
    \end{minipage}%
    \begin{minipage}[t]{0.45\linewidth}
    \centering
    \includegraphics[height=4.5cm,width=7cm]{防晒霜(紫外红外).png}
    \caption{涂防晒霜透射率曲线(紫外红外)}
    \end{minipage}
    \end{figure}

    我们可以把它们各自的有效部分整合在一起,我们将可见光数据的$380nm-750nm$部分替代紫外红外数据的对应部分可以得到合成的曲线。

    \begin{figure}[htbp]
    \begin{minipage}[t]{0.45\linewidth}
    \centering
    \includegraphics[height=4.5cm,width=7cm]{玻璃片合并图.png}
    \caption{玻璃片透射率曲线}
    \end{minipage}%
    \begin{minipage}[t]{0.45\linewidth}
    \centering
    \includegraphics[height=4.5cm,width=7cm]{防晒霜合并图.png}
    \caption{涂防晒霜透射率曲线}
    \end{minipage}
    \end{figure}

    我们主要对比分析二者在紫外部分的情况,我们可以发现玻璃片的透射曲线在紫外部分透射率一致保持在$40\%$以上,而涂上防晒霜之后随着波长的增大透射率在下降,而且下降幅度较大,这说明了防晒霜确确实实起到了隔绝紫外线的作用。

    \subsection{测量溴钨灯的灯丝温度}

    我们将光谱仪的积分时间设置为$0.1ms$,帧速率设置为$100ms$。调整光纤头的角度,使其对准溴钨灯灯丝,使光谱信号最强(超过 50000)。分别将溴钨灯电流设置为 $3.0A$、$2.8A$、$2.7A$、$2.6A$、$2.4A$、$2.2A$、$2.0A$,记录每一次的实际电压和电流值,并保存相应的测量光谱曲线。截取 3$80nm-900nm$ 波长范围的光谱数据。
    利用实验室给出的标准灯在 $2856K $温度下的实际光谱和测量光谱,根据$I(\lambda,T)=I'(\lambda,T)\xi(\lambda)$计算光谱仪在 $380nm-900nm$ 范围内的传递函数$\xi(\lambda)$并利用传递函数计算修正后的钨灯发射光谱。

    我们计算出修正后的钨灯发射光谱后可以利用普朗克黑体辐射公式进行拟合从而得到对应温度$T$,我们得到的测量曲线和拟合曲线如下:

    \begin{figure}[htbp]
    \begin{minipage}[t]{0.45\linewidth}
    \centering
    \includegraphics[height=4.5cm,width=7cm]{2.0A.png}
    \caption{电流设置为2.0A的光谱曲线}
    \end{minipage}%
    \begin{minipage}[t]{0.45\linewidth}
    \centering
    \includegraphics[height=4.5cm,width=7cm]{2.2A.png}
    \caption{电流设置为2.2A的光谱曲线}
    \end{minipage}
    \end{figure}

    \begin{figure}[htbp]
    \begin{minipage}[t]{0.45\linewidth}
    \centering
    \includegraphics[height=4.5cm,width=7cm]{2.4A.png}
    \caption{电流设置为2.4A的光谱曲线}
    \end{minipage}%
    \begin{minipage}[t]{0.45\linewidth}
    \centering
    \includegraphics[height=4.5cm,width=7cm]{2.6A.png}
    \caption{电流设置为2.6A的光谱曲线}
    \end{minipage}
    \end{figure}

    \begin{figure}[htbp]
    \begin{minipage}[t]{0.45\linewidth}
    \centering
    \includegraphics[height=4.5cm,width=7cm]{2.7A.png}
    \caption{电流设置为2.7A的光谱曲线}
    \end{minipage}%
    \begin{minipage}[t]{0.45\linewidth}
    \centering
    \includegraphics[height=4.5cm,width=7cm]{2.8A.png}
    \caption{电流设置为2.8A的光谱曲线}
    \end{minipage}
    \end{figure}

    \begin{figure}[H]
    \centering
    \subfloat{
        \includegraphics[width=0.4\textwidth]{3A.png}
    }
    \caption{电流设置为3.0A的光谱曲线}
    \end{figure}

    我们将各组对应的数据汇总为下表:

    \begin{table}[ht]
    \centering
    \caption{各组对应的功率与温度}
    \begin{tabular}{|c|c|c|c|c|c|}\hline
    设置电流(A)&实际电流(A)&实际电压(V)&功率(W)&温度T(K)&$T^4(K^4)$\\\hline
    2.000  & 2.000  &6.39&12.780& 2449.615&$3.601\times 10^{13}$  \\\hline
    2.200 &2.200&7.60&16.720&2550.297&$4.230\times 10^{13}$\\\hline
    2.400&2.400&8.93&21.432&2675.743&$5.126\times 10^{13}$\\\hline
    2.600&2.600&10.36&26.936&2797.285&$6.123\times 10^{13}$\\\hline
    2.700&2.700&11.07&29.889&2856.752&$6.660\times 10^{13}$\\\hline
    2.800&2.800&11.81&33.068&2914.613&$7.216\times 10^{13}$\\\hline
    3.000&2.807&12.00&33.684&2923.557&$7.305\times 10^{13}$\\\hline
    \end{tabular}%
    \end{table}%

    我们根据上述数据可以对$P$和$T^4$的关系进行拟合,拟合结果为:
    \begin{figure}[H]
    \centering
    \subfloat{
        \includegraphics[width=0.6\textwidth]{直线拟合.png}
    }
    \caption{P与$T^4$的关系图}
    \end{figure}

    我们拟合的结果$R^2=0.99607$非常接近1,因此拟合的效果还是比较好的,这和斯特藩-玻尔兹曼定律是相符的,也说明了我们温度的计算是合理的。不过由于我们的温度是通过拟合曲线得出的,而且$T^4$会放大误差,所以从图像上来看拟合的结果不是太好,但是在实验误差范围内我们是可以认为$P$和$T^4$是成正比的。

\section{分析讨论}

(a)我们在测量元素特征谱线的实验中会发现在测量过程中光强会出现涨落的现象,这可能是因为我们的氢灯、钠灯等是通过脉冲供电的方式进行工作的,而我们采集数据时使用的是积分时间的方式,如果积分时间和脉冲不同步就可能出现实验中出现的光强涨落的现象,但是这个现象也较难避免,因为如果我们积分时间远大于脉冲时间的话会导致光强过大,难以采集。但是我们本实验中关注的是峰值处的波长,因此这一现象并不会产生很大的干扰。

(b)我们在最后的实验是通过曲线拟合的方式获得了温度,我们可以考虑在溴钨灯装置中增加测温装置,这样我们可以得到真实温度,然后与拟合温度进行比较,同时我们用真实温度进行计算可能会和斯特藩-玻尔兹曼定律符合的更好。

\section{原始数据}

    \begin{figure}[H]
    \centering
    \subfloat{
        \includegraphics[width=0.5\textwidth]{光谱实验数据.png}
    }
    \caption{光谱实验数据}
    \end{figure}
\end{document}